\documentclass[11pt,letterpaper]{article}
\usepackage[T1]{fontenc}
\usepackage{times}
\usepackage{epsfig,graphicx,tabularx,color}
\usepackage[cp850]{inputenc}
\usepackage[english]{babel}
\usepackage{cite}
\usepackage{amsmath,amssymb,amsfonts,amsthm,mathrsfs,comment,mathdots,multirow,tikz}
\usepackage{rotating}
\usepackage[f]{esvect}
\usepackage{times}
\usepackage{fancyhdr}
\usepackage{lipsum}

\pagestyle{fancy}

%% Setting up pagestyles for ``fancy''
\lhead{\sl }
\rhead{\sl \thepage}
\lfoot{}
\cfoot{}
\rfoot{}

\setlength{\textheight}{24.13cm}
\setlength{\headheight}{1cm}
\setlength{\headwidth}{17.7cm}
\setlength{\textwidth}{17.7cm}
\setlength{\oddsidemargin}{-0.75cm}
\setlength{\evensidemargin}{-0.75cm}
\setlength{\hoffset}{0.3cm}
\addtolength{\topmargin}{-0.75cm}
\setlength{\voffset}{-2cm}
\setlength{\headsep}{0.5cm}
\setlength{\parskip}{1.3ex}
\setlength{\parindent}{0cm}

\bibliographystyle{unsrt} %plain} %Choose a bibliograhpic style bbrv, plain, unsrt and acm

\newtheorem{theorem}{Theorem}[section]
\newtheorem{corollary}[theorem]{Corollary}
\newtheorem{lemma}[theorem]{Lemma}
\newtheorem{form}[theorem]{Formula}
\newtheorem{algorithm}[theorem]{Algorithm}
\newtheorem{definition}[theorem]{Definition}
\newtheorem{proposition}[theorem]{Proposition}
\newtheorem{example}[theorem]{Example}
\theoremstyle{definition}
\newtheorem{remark}[theorem]{Remark}

\newenvironment{proofsketch}{\paragraph{\it Sketch of Proof.}}{\hfill$\square$}

\def\sphline{\noalign{\vskip3pt}\hline\noalign{\vskip3pt}}

\def\Xint#1{\mathchoice
   {\XXint\displaystyle\textstyle{#1}}%
   {\XXint\textstyle\scriptstyle{#1}}%
   {\XXint\scriptstyle\scriptscriptstyle{#1}}%
   {\XXint\scriptscriptstyle\scriptscriptstyle{#1}}%
   \!\int}
\def\XXint#1#2#3{{\setbox0=\hbox{$#1{#2#3}{\int}$}
     \vcenter{\hbox{$#2#3$}}\kern-.5\wd0}}
\def\ddashint{\Xint=}
\def\dashint{\Xint-}
\def\xint{\Xint{\times}}

\def\ud{{\rm\,d}}
\def\fl{{\rm\,fl}}

\def\C{\mathbb{C}}
\def\D{\mathbb{D}}
\def\F{\mathbb{F}}
\def\I{\mathbb{I}}
\def\K{\mathbb{K}}
\def\N{\mathbb{N}}
\def\P{\mathbb{P}}
\def\Q{\mathbb{Q}}
\def\R{\mathbb{R}}
\def\Sph{\mathbb{S}}
\def\T{\mathbb{T}}
\def\U{\mathbb{U}}
\def\Z{\mathbb{Z}}

\def\BB{\mathcal{B}}
\def\CC{\mathcal{C}}
\def\DD{\mathcal{D}}
\def\FF{\mathcal{F}}
\def\LL{\mathcal{L}}
\def\OO{\mathcal{O}}
\def\PP{\mathcal{P}}
\def\SS{\mathcal{S}}

\def\pr(#1){\left({#1}\right)}
\def\br[#1]{\left[{#1}\right]}
\def\sr#1{\left\{{#1}\right\}}
\def\abs#1{\left|{#1}\right|}
\def\norm#1{\left\|{#1}\right\|}
\def\conj#1{\overline{#1}}

\def\pFq#1#2{{\,}_{#1}F_{#2}}

\def\i{{\rm i}}
\def\for{\hbox{ for }}
\def\qand{\hbox{ and }}
%\def\Jin{J_+^{-1}}
%\def\CC{{\cal C}}
%\def\half{{1 \over 2}}
%\def\E{{\rm e}}
%\def\addtab#1={#1\;&=}
%\def\meeq#1{\def\ccr{\\\addtab}
%\def\qas{\quad\hbox{as}\quad}
%\tabskip=\@centering
% \begin{align*}
% \addtab#1
% \end{align*}
%  }
%\def\rmz{{\rm z}}

\newcommand{\diag}{\operatorname{diag}}
\newcommand{\rank}{\operatorname{rank}}
\newcommand{\mspan}{\operatorname{span}}
\newcommand{\var}{\operatorname{var}}
\newcommand{\VEC}{\operatorname{vec}}
\newcommand{\cond}{\operatorname{cond}}
\newcommand{\sinc}{\operatorname{sinc}}
\newcommand{\logop}{\operatorname{log1p}}
\newcommand{\argmax}{\operatornamewithlimits{arg\,max}}
\newcommand{\argmin}{\operatornamewithlimits{arg\,min}}
\newcommand{\erf}{\operatorname{erf}}

\newcommand{\mathprog}[4]{
\begin{subequations}\label{#4}
\begin{align}
\label{#4-a}
\hbox{minimize } & #1,\\
\label{#4-b}
\hbox{subject to } & #2,\\
\label{#4-c}
\hbox{and } & #3.
\end{align}
\end{subequations}}

\newcommand*\circled[1]{\tikz[baseline=(char.base)]{
            \node[shape=circle,draw,inner sep=2pt] (char) {#1};}}

\def\red{\color{red}}
\def\green{\color{green}}
\def\blue{\color{blue}}

\usepackage{longtable}

\begin{document}
\title{MATH 2160, Chapter 5 Summary \& Exercises}
\author{Richard M. Slevinsky}
\date{}
\maketitle

\section*{A Conversation with Slevinsky}

\begin{longtable}{p{0.475\textwidth}|p{0.475\textwidth}}
\hline
Problems & Solutions\\
\hline
How do I interpolate a periodic function and preserve the periodicity? & There are two scenarios that are related by a variable transformation. If $f$ is a function of a complex variable $z$, then the orthogonal polynomials with respect to $L^2(\U,\ud z)$ are the integer powers $\{z^k\}_{k=-\infty}^{+\infty}$. Both the function $f$ and the basis $z^k$ are periodic on the unit circle $\U$, because $\U$ has no beginning and no end. This allows us to project $f$ onto its {\em Laurent series}:
\[
f(z) \sim \sum_{k=-\infty}^{+\infty} f_k z^k,
\]
where the coefficients are given by:
\[
f_k = \dfrac{1}{2\pi\i}\int_\U \dfrac{f(z)}{z^{k+1}}\ud z.
\]
If $f$ is periodic on $[0,2\pi)$, then we may represent it as a {\em Fourier series}:
\[
f(\theta) \sim \sum_{k=-\infty}^{+\infty} c_k e^{\i k\theta},
\]
where the coefficients are given by:
\[
c_k = \dfrac{1}{2\pi}\int_0^{2\pi} f(\theta) e^{-\i k\theta}\ud\theta.
\]\\
I thought a Fourier series was given in terms of sines and cosines. & Indeed, yet another alternative representation gives:
\[
f(\theta) \sim \dfrac{a_0}{2} + \sum_{k=1}^\infty \left\{ a_k\cos k\theta + b_k\sin k\theta\right\},
\]
where the coefficients are:
\[
a_k = c_k+c_{-k},\quad b_k = \i(c_k-c_{-k}).
\]
\\
\hline
OK, but all these coefficients are given by integrals. How should we approximate them {\em numerically}? & As we saw in our chapter on numerical differentiation and integration, the composite trapezoidal rule worked {\em exceptionally well} on integrands with periodicity. This is precisely because it {\em is the Gaussian quadrature rule} on $L^2(\U,\ud z)$. This leads us to the DFT matrix and its inverse.\\
\hline
What is the DFT matrix? & The {\em discrete Fourier transform} (DFT) matrix represents the transformation from $N$ equispaced function samples on the unit circle to $N$ approximate Maclaurin coefficients $\hat{f}_k$, for $k=0,\ldots,N-1$.\\
\hline
Oh, I remember now!
\[
\FF_N = \begin{bmatrix}
1 & 1 & \cdots & 1\\
1 & \omega & \cdots & \omega^{N-1}\\
\vdots & \vdots & \ddots & \vdots\\
1 & \omega^{N-1} & \cdots & \omega^{(N-1)^2}
\end{bmatrix},
\]
where $\omega = e^{-2\pi\i/N}$. & Exactly! This matrix has {\em so much structure} that when $N$ is a power of two, we may decompose its odd-even permutation into the product of a sparse matrix with $2N$ entries and a matrix containing two copies of the DFT matrix of half the size:
\[
\FF_NP_N = \begin{bmatrix}
I_{N/2} & \Omega_{N/2}\\
I_{N/2} & -\Omega_{N/2}\\
\end{bmatrix}
\begin{bmatrix}
\FF_{N/2} & 0\\
0 & \FF_{N/2}\\
\end{bmatrix}.
\]\\
\hline
Let me see if I've got this right; if $N=4$:
\[
\FF_4 = \begin{bmatrix}
1 & 1 & 1 & 1\\
1 & \omega & \omega^2 & \omega^3\\
1 & \omega^2 & \omega^4 & \omega^6\\
1 & \omega^3 & \omega^6 & \omega^9
\end{bmatrix},
\]
then since $\omega^4 = (e^{-2\pi\i/4})^4\equiv1$:
\[
\FF_4 = \begin{bmatrix}
1 & 1 & 1 & 1\\
1 & \omega & \omega^2 & \omega^3\\
1 & \omega^2 & 1 & \omega^2\\
1 & \omega^3 & \omega^2 & \omega
\end{bmatrix}.
\]
& Yep, and if we swap the second and third columns:
\[
\FF_4[:,[1;3;2;4]] = \begin{bmatrix}
1 & 1 & 1 & 1\\
1 & \omega^2 & \omega & \omega^3\\
1 & 1 & \omega^2 & \omega^2\\
1 & \omega^2 & \omega^3 & \omega
\end{bmatrix},
\]
then we can write:
\[
\FF_4[:,[1;3;2;4]] = \begin{bmatrix}
1 & 0 & 1 & 0\\
0 & 1 & 0 & \omega\\
1 & 0 & -1 & 0\\
0 & 1 & 0 & -\omega\\
\end{bmatrix}
\begin{bmatrix}
\FF_2 & 0\\
0 & \FF_2\\
\end{bmatrix}.
\]\\
\hline
Whoa, that's a lot of omegas. I still don't understand how this gets us the $\OO(N\log N)$ complexity, though. & If we continue the factorization {\em recursively}, then we can visualize it as a {\em tree}. For $N=8$, for example:

\begin{tikzpicture}[level/.style={sibling distance=43mm/#1},grow=up]
\node (a){$\FF_8$}
    child {node (b){$\FF_4$}
        child {node (c){$\FF_2$}
            child {node (d){$\FF_1$}}
	    child {node (e){$\FF_1$}}
	}
	child {node (f){$\FF_2$}
	    child {node (g){$\FF_1$}}
	    child {node (h){$\FF_1$}}
	}
    }
    child {node (i){$\FF_4$}
        child {node (j){$\FF_2$}
            child {node (k){$\FF_1$}}
            child {node (l){$\FF_1$}}
        }
        child {node (m){$\FF_2$}
            child {node (n){$\FF_1$}}
            child {node (o){$\FF_1$}}
        }
    };
\end{tikzpicture}
There are four levels of the tree with eight $1$-by-$1$ matrices at the leaves.\\
\hline
Very interesting. Wait, why are there hats on the coefficients? In which sense are they approximate? & The phenomenon of {\em aliasing} shows us that the $\hat{f}_k$ contain {\em all} the Laurent modes:
\[
\hat{f}_k = f_k + f_{k-N} + f_{k+N} + f_{k-2N} + f_{k+2N}+\cdots+.
\]
Thus, if there is decay in the coefficients and if we use a lot of them, then we will converge to the true Laurent coefficients.\\
All of this technology transfers to Chebyshev polynomials of the first and second kinds, right? & Yep. The DCT and the DST allow us to create polynomial interpolants to $f\in C([-1,1])$ in only $\OO(N\log N)$ operations. Let me tell ya folks, this is ``yuuuuuuuuuuuuuuuuuuuuuuuuuuuuuuuuuuuuuuge.''\\
...making polynomial interpolation great again? & Precisely.\\
We have two polynomial bases $T_k$ and $U_k$. When should we use which one? & It depends what you would like to do, but there is a banded conversion between the bases, so we can change our representation very rapidly. Differentiation and integration is done term-by-term.\\
\hline
\end{longtable}

\section*{Exercises}

\begin{enumerate}

\item If we have the Laurent expansion of a function $f(z)$, then we may easily calculate its $2$-norm only in terms of the coefficients. What is the expression for:
\[
\norm{f}_2^2 = \abs{\int_\U \abs{f(z)}^2\ud z} ?
\]
Alternatively, when $f(\theta)$ is a periodic function on $[0,2\pi)$, what is the expression for:
\[
\norm{f}_2^2 = \int_0^{2\pi} \abs{f(\theta)}^2\ud\theta,\quad \hbox{in terms of $a_k$ and $b_k$, or $c_k$?}
\]

\item Carefully confirm the factorization of $\FF_8P_8$ in terms of $I_4$, $\Omega_4$, and $\FF_4$. What is the inverse of the permutation matrix $P_8$?

\item In problem 4 of chapter 3, we derived the $N(=N(a,\varepsilon))$ required to satisfy $\norm{f-p_N}_\infty \le \varepsilon \norm{f}_\infty$ for $f(x) = (x^2+a^2)^{-1}$. It was:
\[
N(a,\varepsilon) \ge -\log\left(\varepsilon\dfrac{(a+\sqrt{a^2+1})^2-1}{a^2}\right)/\log(a+\sqrt{a^2+1}).
\]
Use the DCT to examine how close the first $N$ coefficients of the interpolant are to the first $N$ coefficients of the expansion. In your opinion, is aliasing a concern for this particular function?

\item Use the conversion matrix:
\[
\CC_N = \begin{bmatrix}
1 & 0 & -\tfrac{1}{2}\\
& \tfrac{1}{2} & 0 & -\tfrac{1}{2}\\
& & \ddots & \ddots & \ddots\\
& & & \tfrac{1}{2} & 0 & -\tfrac{1}{2}\\
& & & & \tfrac{1}{2} & 0\\
& & & & & \tfrac{1}{2}\\
\end{bmatrix},
\]
between the $T_k$ and $U_k$ bases to confirm numerically the exact Chebyshev-$U$ coefficients of $e^x$ in Table 5.3 given the exact Chebyshev-$T$ coefficients in Table 5.2 in terms of the modified Bessel functions of the first kind $I_k(1)$.

\end{enumerate}

\end{document}