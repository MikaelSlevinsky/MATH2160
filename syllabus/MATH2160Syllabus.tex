% from ROASS.SyllabusTemplateSample-1.0-20160713.doc
\documentclass[12pt]{article}

%%%%%%%%%%%%%%%%%%%%%%%%%%%%%%%%%%%%%%%%%%%%%%%%%%%%%%%%%%%%%%%%%%
\newcommand{\UNIVERSITY}{University of Manitoba}
\newcommand{\FACULTY}{Faculty of Science}
\newcommand{\DEPARTMENT}{Department of Mathematics}
\newcommand{\COURSE}{MATH 2160 A01: Numerical Analysis 1}
%%%%%%%%%%%%%%%%%%%%%%%%%%%%%%%%%%%%%%%%%%%%%%%%%%%%%%%%%%%%%%%%%%



\usepackage[hmargin=2cm,
                vmargin=2cm,
                letterpaper]{geometry}
\usepackage[colorlinks=true,	% use false if for printing black & white
            pdfstartview=FitV,
            linkcolor=blue,
            %citecolor=black,
            %urlcolor=black,
            bookmarks=false,
            bookmarksopen=false,
            backref=false,
            pdfpagelabels=false]{hyperref}
\usepackage{fancyhdr}
\usepackage{lastpage}
\usepackage{charter}
\usepackage{longtable}
\usepackage{booktabs}
\usepackage{enumerate}
\usepackage{graphicx}
\usepackage{amsmath}
\usepackage{amssymb}
\usepackage{varioref}
\usepackage[export]{adjustbox}	% allows valign=t in \includegraphics
%\usepackage{datenumber}
%\usepackage{ifthen}
%\usepackage{boolexpr}
%\usepackage{pdftexcmds}
%\usepackage{xstring}

\newlength{\infoTableSpace}
\setlength{\infoTableSpace}{1em plus 1em minus 0.5em}


\lhead{\COURSE}
\chead{}
\rhead{Page \thepage}
\lfoot{}
\cfoot{}
\rfoot{}
\pagestyle{fancy}



\begin{document}

\hspace*{-\tabcolsep}%
\begin{tabular}{l@{\hspace{0.15\linewidth}}l}
\includegraphics[height=10ex,valign=t]{uofm} &
\begin{minipage}[t]{0.5\textwidth}
	\bf\large
	
	\UNIVERSITY
	
	\FACULTY
	
	\DEPARTMENT
\end{minipage}
\end{tabular}



%\tableofcontents


\section{Course Details}

\noindent
\begin{tabular*}{\linewidth}{r@{\hspace{\infoTableSpace}}l}
	\toprule
	\textbf{Course Title \& Number} 				& \COURSE	\\
	\textbf{Number of Credit Hours} 				& 3	\\
	\textbf{Lecture Times \& Days of Week} 			& 8:30 am\,--\, 9:45 am TR 	\\
	\textbf{Location for lectures}	&  BULLER 527 \\ %ARMES 201	\\
	\textbf{Lab Times \& Days of Week} 			& 4:30 pm\,--\, 5:20 pm W 	\\
	\textbf{Location for labs}	&  B01: BULLER 315, B02: BULLER 306\\
	\textbf{Pre-Requisites} 						& [MATH 1232 (C) or MATH 1690 (C) or MATH 1700 (B)\\
	 & or MATH 1701 (B) or MATH 1710 (B)] and \\
	 & [MATH 1220 (C) or MATH 1300 (B) or MATH 1301 (B)] \\
	 \textbf{Not to be held with} & MATH 2120 (Eng) \\
	\bottomrule
\end{tabular*}



\section{Instructor Contact Information}

\noindent
\begin{tabular*}{\linewidth}{r@{\hspace{\infoTableSpace}}l}
	\toprule
	\textbf{Instructor(s) Name} 			& Richard Mika\"el Slevinsky 	\\ 
	%\textbf{Preferred Form of Address} 		& 	\\
	\textbf{Office Location}				& 421 Machray Hall	\\
	\textbf{Office Hours or Availability} & You are welcome any time I am in my office \\
	\textbf{Office Phone Number} 			& 204-474-6647	\\
	\textbf{Cell Phone Number} 			& 431-334-9330 (mathematical emergencies) \\
	\textbf{E-mail}	& \begin{minipage}[t]{0.65\linewidth}	
			\href{mailto:Richard.Slevinsky@umanitoba.ca}{\texttt{Richard.Slevinsky@umanitoba.ca}}
			All e-mail communication must conform to the 
			\href{http://umanitoba.ca/admin/governance/media/Electronic_Communication_with_Students_Policy_-_2014_06_05.pdf}{Communicating with Students} university policy. E-mail inquiries will be answered as soon as possible.			
		\end{minipage} \\
	\bottomrule
\end{tabular*}


\section{Required material}

The course is based on course notes that are available through the bookstore and electronically on UM Learn.

A copy of {\sc Julia-0.6}, available from \href{https://julialang.org/downloads/}{https://julialang.org/downloads/}, under the section {\em Julia (command line version)}.

\section{Course Outline}

Elementary techniques of numerical solution of mathematical problems: solution of equations, linear systems of equations, nonlinear equations; finite and divided differences, interpolation; numerical differentiation and integration.

\section{Attendance Policy}

Attendance in the lectures and labs is mandatory.

\section{Course Evaluation Methods}

Students will be assessed using assignments, lab quizzes, a midterm, and a final examination.

\begin{center}
\begin{tabular}[t]{p{0.3\linewidth}p{0.4\linewidth}p{0.2\linewidth}}
\toprule
\mbox{}\newline \textbf{Date} &	\mbox{}\newline\textbf{Assessment Tool}	& \textbf{Value of \newline Final Grade} \\
\midrule
By chapter & Assignments & 20\% \\
Bi-weekly & Lab quizzes & 10\% \\
Thursday, October 26, 2016 & In-class midterm examination & 25\% \\
TBA & Final examination & 45\% \\
\bottomrule
\end{tabular}
\end{center}

\section{Grading}

We will use the following scheme, subject to adjustments due to overall class performance.

\begin{center}
\begin{tabular}[t]{lcc}
\toprule
Letter Grade & Minimum percentage to guarantee & Final Grade Point  \\
\midrule
A+ & 95 & 4.5 \\
A & 86 & 4.0 \\
B+ & 80& 3.5 \\
B & 72 &  3.0 \\
C+ & 65  & 2.5 \\
C & 60 & 2.0 \\
D & 50& 1.0  \\
\bottomrule
\end{tabular}
\end{center}

\section{Assignment Grading Times}

Normally, students can expect assignments and quizzes graded and returned within one week. Feedback will include a numerical grade and any comments that may be helpful for the students and/or that are required for the justification of the grade.

\section{Schedule of tests and quizzes and assignments}

The quizzes will be held in the labs bi-weekly; the assignments for each chapter of the course notes are tentatively due as follows.

\begin{center}
\begin{tabular}[t]{rr}
\toprule
\textbf{Assignment} & \textbf{Date}\\
\midrule
Chapter 1 & 5:00 pm, Friday, September 29, 2017\\
Chapter 2 & 5:00 pm, Friday, October 20, 2017\\
Chapter 3 & 5:00 pm, Friday, November 3, 2017\\
Chapter 4 & 5:00 pm, Friday, November 17, 2017\\
Chapter 5 & 5:00 pm, Friday, December 1, 2017\\
Chapter 6 & 5:00 pm, Friday, December 8, 2017\\
\bottomrule
\end{tabular}
\end{center}

\section{Policy on missed or late assignments, quizzes, test}

No late assignments will be accepted; no missed quizzes will be accepted. However, the lowest quiz and the lowest assignment grades will not be counted.

A student missing the midterm who contacts the instructor within two days of the scheduled midterm with supporting documentation may be provided with the procedure for a make-up midterm.

\section{Course Technology}

It is the general University of Manitoba policy that all technology resources are to be used in a responsible, efficient, ethical and legal manner. The student can use all technology in classroom setting only for educational purposes approved by instructor and/or the University of Manitoba Student Accessibility Services. Student should not participate in personal direct electronic messaging / posting activities (e-mail, texting, video or voice chat, wikis, blogs, social networking (e.g. Facebook) online and offline ``gaming'' during scheduled class time.

We will use UM Learn and e-mail communication as the primary technologies and transmission of information.

\section{Recording Class Lectures}

Richard Mika\"el Slevinsky holds copyright over the course material, presentations, and lectures which form part of this course.  Audio or video recording of lectures, labs, and/or presentations is allowed in any format, openly or surreptitiously, in whole or in part so long as they are only used for the participant's private study and research.

\section{Student Accessibility Services}

If you are a student with a disability, please contact SAS for academic accommodation supports and services such as note-taking, interpreting, assistive technology and exam accommodations. Students who have, or think they may have, a disability (e.g. mental illness, learning, medical, hearing, injury-related, visual) are invited to contact SAS to arrange a confidential consultation. 

\begin{quote}
Student Accessibility Services \url{http://umanitoba.ca/student/saa/accessibility/} \\
520 University Centre \\
204 474 7423 \\
\href{mailto:Student_accessibility@umanitoba.ca}{\texttt{Student\_accessibility@umanitoba.ca}}
\end{quote}

\section{Academic Integrity}

This section provides additional information to the general information about academic integrity and student discipline described in Schedule A Policies and Resources.

\begin{enumerate}[(i)]
\item Exams and quizzes are to be completed independently.

\item Students are encouraged to complete assignments in collaboration with other students registered in MATH 2160. Students must clearly report collaboration on the assignment as, for example:

``Questions 1 \& 2 were completed in collaboration with Jane Doe, Student ID \#.''

Students should be able to complete more than 65\% of any given assignment independently.
\end{enumerate}

\end{document}