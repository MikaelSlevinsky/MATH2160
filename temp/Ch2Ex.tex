\documentclass[11pt,letterpaper]{article}
\usepackage[T1]{fontenc}
\usepackage{times}
\usepackage{epsfig,graphicx,tabularx,color}
\usepackage[cp850]{inputenc}
\usepackage[english]{babel}
\usepackage{cite}
\usepackage{amsmath,amssymb,amsfonts,amsthm,mathrsfs,comment,mathdots,multirow,tikz}
\usepackage{rotating}
\usepackage[f]{esvect}
\usepackage{times}
\usepackage{fancyhdr}
\usepackage{lipsum}

\graphicspath{{../figures/}}

\pagestyle{fancy}

%% Setting up pagestyles for ``fancy''
\lhead{\sl }
\rhead{\sl \thepage}
\lfoot{}
\cfoot{}
\rfoot{}

\setlength{\textheight}{24.13cm}
\setlength{\headheight}{1cm}
\setlength{\headwidth}{17.7cm}
\setlength{\textwidth}{17.7cm}
\setlength{\oddsidemargin}{-0.75cm}
\setlength{\evensidemargin}{-0.75cm}
\setlength{\hoffset}{0.3cm}
\addtolength{\topmargin}{-0.75cm}
\setlength{\voffset}{-2cm}
\setlength{\headsep}{0.5cm}
\setlength{\parskip}{1.3ex}
\setlength{\parindent}{0cm}

\bibliographystyle{unsrt} %plain} %Choose a bibliograhpic style bbrv, plain, unsrt and acm

\newtheorem{theorem}{Theorem}[section]
\newtheorem{corollary}[theorem]{Corollary}
\newtheorem{lemma}[theorem]{Lemma}
\newtheorem{form}[theorem]{Formula}
\newtheorem{algorithm}[theorem]{Algorithm}
\newtheorem{definition}[theorem]{Definition}
\newtheorem{proposition}[theorem]{Proposition}
\newtheorem{example}[theorem]{Example}
\theoremstyle{definition}
\newtheorem{remark}[theorem]{Remark}

\newenvironment{proofsketch}{\paragraph{\it Sketch of Proof.}}{\hfill$\square$}

\def\sphline{\noalign{\vskip3pt}\hline\noalign{\vskip3pt}}

\def\Xint#1{\mathchoice
   {\XXint\displaystyle\textstyle{#1}}%
   {\XXint\textstyle\scriptstyle{#1}}%
   {\XXint\scriptstyle\scriptscriptstyle{#1}}%
   {\XXint\scriptscriptstyle\scriptscriptstyle{#1}}%
   \!\int}
\def\XXint#1#2#3{{\setbox0=\hbox{$#1{#2#3}{\int}$}
     \vcenter{\hbox{$#2#3$}}\kern-.5\wd0}}
\def\ddashint{\Xint=}
\def\dashint{\Xint-}
\def\xint{\Xint{\times}}

\def\ud{{\rm\,d}}
\def\fl{{\rm\,fl}}

\def\C{\mathbb{C}}
\def\D{\mathbb{D}}
\def\F{\mathbb{F}}
\def\I{\mathbb{I}}
\def\K{\mathbb{K}}
\def\N{\mathbb{N}}
\def\P{\mathbb{P}}
\def\Q{\mathbb{Q}}
\def\R{\mathbb{R}}
\def\Sph{\mathbb{S}}
\def\T{\mathbb{T}}
\def\U{\mathbb{U}}
\def\Z{\mathbb{Z}}

\def\BB{\mathcal{B}}
\def\CC{\mathcal{C}}
\def\DD{\mathcal{D}}
\def\FF{\mathcal{F}}
\def\LL{\mathcal{L}}
\def\OO{\mathcal{O}}
\def\PP{\mathcal{P}}
\def\SS{\mathcal{S}}

\def\pr(#1){\left({#1}\right)}
\def\br[#1]{\left[{#1}\right]}
\def\sr#1{\left\{{#1}\right\}}
\def\abs#1{\left|{#1}\right|}
\def\norm#1{\left\|{#1}\right\|}
\def\conj#1{\overline{#1}}

\def\pFq#1#2{{\,}_{#1}F_{#2}}

\def\i{{\rm i}}
\def\for{\hbox{ for }}
\def\qand{\hbox{ and }}
%\def\Jin{J_+^{-1}}
%\def\CC{{\cal C}}
%\def\half{{1 \over 2}}
%\def\E{{\rm e}}
%\def\addtab#1={#1\;&=}
%\def\meeq#1{\def\ccr{\\\addtab}
%\def\qas{\quad\hbox{as}\quad}
%\tabskip=\@centering
% \begin{align*}
% \addtab#1
% \end{align*}
%  }
%\def\rmz{{\rm z}}

\newcommand{\diag}{\operatorname{diag}}
\newcommand{\dist}{\operatorname{dist}}
\newcommand{\diam}{\operatorname{diam}}
\newcommand{\rank}{\operatorname{rank}}
\newcommand{\mspan}{\operatorname{span}}
\newcommand{\var}{\operatorname{var}}
\newcommand{\VEC}{\operatorname{vec}}
\newcommand{\cond}{\operatorname{cond}}
\newcommand{\sinc}{\operatorname{sinc}}
\newcommand{\logop}{\operatorname{log1p}}
\newcommand{\argmax}{\operatornamewithlimits{arg\,max}}
\newcommand{\argmin}{\operatornamewithlimits{arg\,min}}
\newcommand{\erf}{\operatorname{erf}}

\newcommand{\mathprog}[4]{
\begin{subequations}\label{#4}
\begin{align}
\label{#4-a}
\hbox{minimize } & #1,\\
\label{#4-b}
\hbox{subject to } & #2,\\
\label{#4-c}
\hbox{and } & #3.
\end{align}
\end{subequations}}

\newcommand*\circled[1]{\tikz[baseline=(char.base)]{
            \node[shape=circle,draw,inner sep=2pt] (char) {#1};}}

\def\red{\color{red}}
\def\green{\color{green}}
\def\blue{\color{blue}}


\usepackage{longtable}

\begin{document}
\title{MATH 2160, Chapter 2 Summary \& Exercises}
\author{Richard M. Slevinsky}
\date{}
\maketitle

\section*{A Conversation with Slevinsky}

\begin{longtable}{p{0.475\textwidth}|p{0.475\textwidth}}
\hline
Problems & Solutions\\
\hline
$Ax = b$, $A$ square. & A good algorithm is Gaussian elimination. It is a direct algorithm that terminates after $\OO(n^3)$ operations. Partial pivoting ensures that it is stable; however, there are corner cases where the rounding errors can accumulate {\em geometrically} with the problem dimension.\\
$Ax = b$, $A$ square and multiple RHS. & First, compute a matrix factorization, such as $LUP$ or $QR$. Each of these costs $\OO(n^3)$ operations, but solution of linear systems with factorizations consisting of triangular or orthogonal matrices costs only $\OO(n^2)$ operations.\\
$Ax = b$, $A$ rectangular with more rows than columns. & This is a least-squares problem. DO NOT SOLVE $A^* A x = A^* b$. Instead, use a reduced $QR$ factorization, where $Q$ is now a rectangular matrix with orthonormal columns and $R$ is still square and upper triangular.\\
$Ax = b$, $A$ rectangular with more columns than rows. & Focus! We didn't study this! This is an ill-posed problem, but it is useful in image compression.\\
$A = V\Lambda V^{-1}$? & Generically, a matrix is not guaranteed to have a spectral decomposition.\\
Fine, what about when $A\in\R^{n\times n}$ is symmetric? & Yes! Even better, the eigenvectors can be chosen to be orthonormal: $A = Q\Lambda Q^\top$.\\
$A = U\Sigma V^*$? & Yes! Every matrix $A\in\C^{m\times n}$ has a singular value decomposition.\\
Cool, but why is this useful? & For one, we now know how to calculate the matrix $2$-norm, since $\norm{A}_2 = \sigma_1$, its largest singular value. For another, if we just take the first $r$ columns of $U$ and $V$ and the $r\times r$ principal submatrix of $\Sigma$, we have the best rank-$r$ approximation to $A$, which is another useful matrix compression technique.\\
What happens when $A$ is large? & The main lessons of this chapter are to take advantage of structure of your linear system; structure usually transpires from the problem you are trying to solve. Important structures are symmetry, sparsity patterns, and definiteness, which are all useful when solving linear systems iteratively.\\
\hline
\end{longtable}

\section*{Exercises}

\begin{enumerate}

\item What can we say about the eigenvalues of a unitary matrix?

\item Determine the SVDs of the following matrices (by hand):
\[
\begin{bmatrix} 3 & 0\\ 0 & -2\end{bmatrix},\qquad \begin{bmatrix} 2 & 0\\ 0 & 0\\ 0 & 3\end{bmatrix},\qquad \begin{bmatrix} 1 & 1\\1 & 1\end{bmatrix},\quad{\rm and}\quad \begin{bmatrix} 1 & 2\\2 & 3\end{bmatrix}.
\]

\item Let $\rho(A)$ denote the {\em spectral radius} of $A\in\C^{n\times n}$, i.e. the largest eigenvalue in absolute value $\abs{\lambda}$. Let $\norm{\cdot}_p$ denote the $p$-norm on $\C^n$ and the induced matrix norm on $\C^{n\times n}$. Show that $\rho(A) \le \norm{A}_p$ for every $1\le p\le\infty$.

\item Suppose $A\in\R^{m\times n}$ and $B\in\R^{n\times m}$ is the matrix obtained by rotating $A$ clockwise $90^{\circ}$. Do $A$ and $B$ have the same singular values? Prove that the answer is yes or give a counterexample.

\item The matrix:
\[
A = \frac{1}{2}\begin{bmatrix} 1 & 1\\ & \ddots & \ddots\\ & & 1 & 1\\ 1 & & & 1\\\end{bmatrix} = \frac{1}{2}(I_n+S_n)\in\R^{n\times n}
\]
represents the averaging of the coordinates of an $n$-sided polygon in a plane. Here, $I_n$ is the identity matrix and $S_n$ is the right-circular shift matrix. Although $A$ is not symmetric, it does have a spectral decomposition. Can you find it?

\item Consider the matrix:
\[
A = \begin{bmatrix} 1 & 2\\ & \ddots & \ddots\\ & & 1 & 2\\ & & & 1\end{bmatrix} \in \R^{n\times n}.
\]
\begin{enumerate}
\item What are the eigenvalues and determinant of $A$?
\item What is $A^{-1}$?
\item Give a nontrivial bounds on $\sigma_1$ and $\sigma_n$, the first and last singular values of $A$. Use {\sc Julia} to build your intuition on the problem, but the bounds should be derived analytically.
\end{enumerate}

\end{enumerate}

\end{document}